
\documentclass[12pt,a4paper]{article} 

\usepackage{float,times,graphicx,mathtools}
\usepackage{amsmath}
\usepackage{amsfonts}
\usepackage{amssymb}
\usepackage{latexsym}
\usepackage{epsfig}
\usepackage{graphicx}
\usepackage{caption}
\usepackage{subcaption}
\usepackage{color}
\usepackage{pdfpages}
\usepackage{natbib}
\usepackage[space]{grffile}
\usepackage{wrapfig}
\usepackage{subcaption}
\usepackage{url}
\usepackage{bbm}

\DeclareMathOperator{\logit}{logit}
\DeclareMathOperator{\tr}{tr}
\bibpunct[, ]{(}{)}{;}{a}{,}{,}
\graphicspath{{../}}  
\addtolength{\oddsidemargin}{-1in}
	\addtolength{\evensidemargin}{-1in}
	\addtolength{\textwidth}{1.75in}
	\addtolength{\topmargin}{-1.3in}
	\addtolength{\textheight}{2in}
\date{\vspace{-5ex}}
\begin{document}

\paragraph{Conversion from period $m_{x}$ to $s_{x}$} \\~\\
Given a period life table with column $m_x$, by assuming UDD, obtain $q_x$ and the corresponding $l_x$ and $d_x$.
\\~\\
$s_x$, the survivorship between age groups, are then calculated by $\frac{L_{x+1}}{L_{x}}$, where $L_x$ denotes the exposures in the period at age $[x, x+1)$. Under UDD, $L_x = l_x - \frac{1}{2} d_x$,

\begin{align*}
s_x &= \frac{L_{x+1}}{L_x} \\~\\
	&\stackrel{\smash{\scriptscriptstyle\mathrm{UDD}}}{=} \frac{l_{x+1} - \frac{1}{2} d_{x+1}}{l_x - \frac{1}{2} d_x} \\~\\
	&= \frac{l_{x+1} (1 - 0.5 \, q_{x+1})}{l_x(1 - 0.5 \, q_x)} \\~\\
	&= \frac{p_x}{(1 - 0.5 \, q_x)} (1 - 0.5 \, q_{x+1}) \\~\\
	&\stackrel{\smash{\scriptscriptstyle\mathrm{UDD}}}{=} {_{0.5}p_{x+0.5}} \quad _{0.5}p_{x+1} \\~\\\ 
	&= p_{x+0.5}
\end{align*}

Therefore under UDD, $s_x$ coincides with the 1 year surviving probability of an individual aged exactly $x+0.5$ (mid-age to mid-age survival). Intuitively, the age group experiences the latter half of the mortality in the currently age group and the early half of the mortality in the next age group.

Under UDD, we also have $_tq_x = t \, q_x \quad \forall t \in [0,1]$ and $f(t) = q_x = {_tp_x} \, \mu_{x+t} \quad \forall t \in [0,1]$. By approximating $\mu_{x+\frac{1}{2}} = m_x$, we have

\begin{align*}
q_x &=  {_{0.5}p_x} \, \, \mu_{x+0.5} \\~\\
	&\approx {_{0.5}p_x} \, \, m_x \\~\\
\implies  \phantom{q} m_x &= \frac{q_x}{1-0.5 \, q_x} \\~\\
\implies  \phantom{m} q_x &= \frac{m_x}{1+0.5 \, m_x} \\~\\
\implies  \phantom{m} s_x &= \frac{1 - 0.5 \, m_x}{1 + 0.5 \, m_{x+1}}
\end{align*}

For survival into the first age group (newborns to mid year) and surviving in the last age group (open age group survival), I used $s_{0^{-}} = 1 - 0.65 \, q_0 = \frac{1}{1+ 0.65 \, m_0}$ and $s_N = q_N = \frac{m_N}{1+0.5 \, m_N}$.

\newpage
Generalising to the 5x5 age-period groups,

\begin{align*}
_5q_x &= \frac{5 \, m_x}{1+ 2.5 \, m_x}
\end{align*}

and
\begin{align*}
 _5s_x &=  \frac{1 - 2.5 \, m_x}{1 + 2.5 \, m_{x+1}}
\end{align*}

Not sure how good this is but it is what I am doing for 5x5.

For survival into the first age group (newborns to mid year) and surviving in the last age group (open age group survival) in the 5x5 case, I used $s_{0^{-}} = 1 - 0.5 \, q_0 = \frac{1}{1+ 0.5 \, m_0}$ and $s_N = q_N = \frac{m_N}{1+0.5 \, m_N}$ (i.e. no uneven distribution of deaths in the first age group).
 
\paragraph{LogQuad Model} \\~\\
Obtained coefficients from the package \textit{MortCast} which contains the coefficients of the LogQuad Model at age groups $0, 5\text{-}9, 10\text{-}14, \dots, 110\text{+}$. Age group $1\text{-}4$ are left out as in the original LQ Model the $_4q_1$ are calibrated such that the modelled $p_0 \, _4p_1$ match the input $_5q_0$. 

In my application to the 5x5 CCMPP I extracted coefficients starting at age group $5\text{-}9$ and set the log mortality of the first age group $0\text{-}4$ to be $h$, the parameter to be estimated (is that okay?).

The population count data for Burkina Faso are in 5-year age groups starting at age 0 up to an open age group $80\text{+}$. Therefore, the estimated mortality rates from the LogQuad Model at age groups $80\text{-}84, 85\text{-}89, \dots, 110\text{+}$ have to be averaged out to obtain mortality rates for $80\text{+}$.

\begin{align*}
m_{80\text{+}} &= \frac{\sum_{i \in \{80\text{-}84, 85\text{-}89, \dots, 110\text{+} \}} w_i m_i }{\sum_{i \in \{80\text{-}84, 85\text{-}89, \dots, 110\text{+} \}} w_i}
\end{align*}
where, under UDD,
\begin{align*}
w_i &= (1-0.5 \, q_i) \prod_{j \in \{80\text{-}84, 85\text{-}89, \dots, 110\text{+} \} \vert j < i} p_j
\end{align*}
Essentially using the exposures $L_x$ as the weights.

Interpolation to single-year age group from $0\text{-}80\text{+}$?
\end{document} 