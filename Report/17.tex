\documentclass[12pt,a4paper]{article} 

\usepackage{float,times,graphicx,mathtools}
\usepackage{amsmath}
\usepackage{amsfonts}
\usepackage{amssymb}
\usepackage{latexsym}
\usepackage{epsfig}
\usepackage{graphicx}
\usepackage{caption}
\usepackage{subcaption}
\usepackage{color}
\usepackage{pdfpages}
\usepackage{natbib}
\usepackage[space]{grffile}
\usepackage{wrapfig}
\usepackage{subcaption}
\usepackage{url}
\usepackage{bbm}
\usepackage{tikzsymbols}

\DeclareMathOperator{\logit}{logit}
\DeclareMathOperator{\tr}{tr}
\bibpunct[, ]{(}{)}{;}{a}{,}{,}
\graphicspath{{../Burkina Faso/12/}}  
\addtolength{\oddsidemargin}{-1in}
	\addtolength{\evensidemargin}{-1in}
	\addtolength{\textwidth}{1.75in}
	\addtolength{\topmargin}{-1.3in}
	\addtolength{\textheight}{2in}
\date{\vspace{-5ex}}

\begin{document}

\begin{itemize}
\item For child and old age component, spline coefficients are given the usual 2nd order penalties $+$ shrinkage towards 0 around the LQ derived priors

\item For the hump component, as discussed last week, want:
	\begin{itemize}
	\item[--] want smooth trends defined by 2nd order differences
	\item[--] does not want the linear trend to continue in periods without sufficient data
	\item[--] want extrapolation of the parameters into periods without data to converge approximately at the current levels
	\item[--] did not use an AR process in the end, but an ARIMA(1,1,0), below is my thought process
	\item[\hookrightarrow] the 2nd order smoothness penalty implies
	\begin{align*}
	\beta_{t+1} &= 2 \beta_t - \beta_{t-1} + \varepsilon_{t+1} \\
	&= \beta_t \, + \underbrace{(\beta_t - \beta_{t-1})}_{\text{continuation of the trend}} + \varepsilon_{t+1}
	\end{align*}
	introduce $\rho \in (0,1)$ such that $\beta_{t+1} &= \beta_t + \rho(\beta_t - \beta_{t-1}) + \varepsilon_{t+1}$, i.e. the next beta does not continue the current trend entirely
	\begin{itemize}
	\item[\cdot] when $\rho \to 0$, a RW1 is obtained, so future coefficients maintain the current level (flat line)
	\item[\cdot] when $\rho \to 1$, a RW2 is obtained, so future coefficients maintain the current trend (linear line)
	\item[\cdot] varying $\rho \in (0,1)$ maintains a trade of between smoothness in the linear trend sense and staying approximately at the current level
	\item[\cdot] \textcolor{red}{I think} this is equivalent to having $\lambda_1 (\boldsymbol{D_1'D_1}) + \lambda_2 (\boldsymbol{D_2'D_2})$ as penalty, where $\boldsymbol{D_i}$ is the $i$-th order difference matrix, based on previous work on the ICAR representation of the spline coefficients
	\item[\cdot] this suggests using an ARIMA(1,1,0) on the coefficients with $\rho$ restricted to be in (0,1), but in practice I used $\rho \in (-1, 1)$ to be consistent with time-series, might not be logical in the smoothing penalty sense? Should I restrict them to be in (0, 1)?
	\end{itemize}
\item[--] will change smoothness priors on gx to be 2nd order diff
\item[--] flatter priors on smoothness penalties? 
\item[--] male B under-smoothed
\item[--] will set males and females lambda to be the same
\item[--] investigating how to set informative priors on the smoothness penalties, as these are difficult to elicit due to the dependency on knot spacing, number of basis, design matrix etc.
\item[--] converged for most countries, except for Gambia (outder mgc exploding), Lesotho (only converged when $\rho$ restricted to (0,1)), etc.
	\end{itemize}
\end{itemize}

\section*{PC Priors}
For each parameter of the child and old age components, priors are derived based on the marginal precision (variance). E.g., $\phi_t$ is approximately the log of mortality rate at age $2$ (assuming other components have very little effect at this age)in year $t$, then we model

\begin{align*}
\boldsymbol{\phi}_t = \boldsymbol{\tilde{\phi}}_t + \boldsymbol{B\beta}
\end{align*} 

 where $\boldsymbol{\tilde{\phi}}_t$ is the vector of estimates derived from the IGME estimates using the LogQuad model and $\boldsymbol{B\beta}$ is a P-spline. Contrary to the usual 2nd order difference penalty on the coefficients, here we do not expect any linear or constant deviation from the IGME estimates, hence a `0`-th order difference penalty is used, where the limiting curve is constant at 0, which is simply a MVN.

\begin{align*}
\boldsymbol{\beta} \sim N(0, \tau^{-1} \boldsymbol{I})
\end{align*} 
Therefore 
\begin{align*}
Var(\boldsymbol{\phi}_t) = Var(\boldsymbol{B\beta}) = \tau^{-1} \boldsymbol{BB'}
\end{align*} 
Due to the even knot spacing, the B-spline basis functions are uniform, hence the diagonal elements of \boldsymbol{BB'} are very similar. In my applications are, the diagonal elements are around 0.5, therefore the marginal variance for $ \sigma^* = VAR(\boldsymbol{\phi}_t) \approx  0.5 \, \tau^{-1}$. Following Simpson et al. (2014), we choose an interval [-U , U] such that Prob($1/ \sqrt{\tau} = \sqrt{\sigma^*} > U$) $= a$ for some small $a$, i.e. we choose a value U as the plausible maximum marginal standard deviation, such that any values larger than U are deemed implausible (with probability $a$). Here the marginal standard deviation is defined on the log deviation from the IGME estimates, which should be relatively small if we believe that the IGME estimates are accurate. In practice, I first chose an interval [-0.5, 0.5] that $\boldsymbol{B\beta}$ is likely to lie in, this can be translate to a percentage error of [0.6, 1.64] (too wide?) of the IGME estimates. Then I obtained $0.31 \, U = 0.5/1.96$ and following Simpson et al. (2014), give the marginal precision $\sigma^*^{-1}$ a Gumbel Type 2 prior with parameters 1/2 and decay rate $\lambda = - \log(0.01)/U$. 

This is done on each parameter independently.
\begin{itemize}
\item[--] what should be the confidence range of the other parameters such as the slope of the child component
\item[--] tired to consider them jointly, however the non-linear structure of the model makes it very difficult to proceed
\item[--] tried to consider priors other than iid MVN for the spline coefficients, as MVN does not take into account of the smoothness, but I still have not worked out the kinks yet, do you think this is necessary? (e.g. AR(2) with $\rho_1 = 2 \alpha$ and $\rho_2 = -\alpha$ with $\alpha \in (0,1)$)
\item[--] still working out ways to set priors for the hump components, as the parameters are slightly more complicated since I want an ARIMA process on the coefficients.
\item[--] will do the same for $g_x$ and $f_x$
\item[--] hyperpriors for variance of the data likelihood remains as Inverse Gamma?
\end{itemize}

\end{document}